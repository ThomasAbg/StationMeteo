\chapter{ Capteur d'Altitude}

\noindent Pour le projet, ayant d\'{e}j\`{a} fait le capteur d\'{e}finie dans les consignes du projet, je met en place le capteur MS5607-02BA03.


\section{ Mise en place du Capteur }

\noindent Le Capteur est disponible dans deux protocoles~: 

\begin{enumerate}
\item  SPI

\item  I2C
\end{enumerate}

\noindent Par d\'{e}faut le capteur est en I2C, on peut passer le Capteur en SPI en dessoudant quelques r\'{e}sistances. 

\noindent D'apr\`{e}s les donn\'{e}es constructrices l'adresse I2C permettant la communication avec le module est 0xEC en lecture ou 0XED en \'{e}criture.

\noindent 

\begin{tabular}{|p{2.3in}|p{1.9in}|} \hline 
Adresse I2C & R\'{e}ponse du Capteur \\ \hline 
0xEE & Renvoie des valeurs (trouv\'{e}es pendant les test) \\ \hline 
0x76 in Read\newline 0x77 in Write & Renvoie Aucune Valeur (D\'{e}finie dans la datasheet) \\ \hline 
\end{tabular}



\noindent 
\subsection{Commande }

\noindent 

\noindent Toutes les commandes suivantes sont disponibles dans la datasheet MS5607-02BA03

\begin{tabular}{|p{2.1in}|p{1.8in}|} \hline 
Nom de la commande & Valeur Hexad\'{e}cimale \\ \hline 
Reset & 0x1E \\ \hline 
Convert D1 256 & 0x40 \\ \hline 
Convert D1 512 & 0x42 \\ \hline 
Convert D1 1024 & 0x44 \\ \hline 
Convert D1 2048 & 0x46 \\ \hline 
Convert D1 4096 & 0x48 \\ \hline 
Convert D2 256 & 0x50 \\ \hline 
Convert D2 512 & 0x52 \\ \hline 
Convert D2 1024 & 0x54 \\ \hline 
Convert D2 2048 & 0x56 \\ \hline 
Convert D2 4096 & 0x58 \\ \hline 
ADC Read & 0x00 \\ \hline 
Prom Read & 0xA0 to 0xAE \\ \hline 
\end{tabular}

\textit{Tableau 1}

\noindent \eject 

\noindent Le capteur fonctionne avec ces commandes.

\begin{enumerate}
\item  D1 : repr\'{e}sente la donn\'{e}e pression

\item  D2 : repr\'{e}sente la donn\'{e}e temp\'{e}rature
\end{enumerate}

\noindent Tous deux peuvent avoir une r\'{e}solution de :256

\begin{enumerate}
\item  512

\item  1024

\item  2048

\item  4096
\end{enumerate}

\noindent La r\'{e}solution change le temps de calcul du capteur, selon l'utilisation nous pouvons avoir plus ou moins besoin de r\'{e}solution grande. Dans notre cas pas utile, le capteur sera juste une indication pour l'utilisateur.

\noindent Remise \`{a} z\'{e}ro : 0x1E

\noindent ADCRead permet de dire au capteur que nous voulons la donn\'{e}e pression-temp\'{e}rature.

\noindent PromRead permet de r\'{e}cup\'{e}rer des coefficients pour affiner et calibrer les donn\'{e}es re\c{c}ues du capteur. Il faut juste appliquer les calculs indiqu\'{e}s sur les donn\'{e}es re\c{c}ues.\underbar{}

\noindent 
\subsection{Trames d'envoy\'{e} et r\'{e}ception}

\noindent 

\noindent Dans chaque trame I2C, il faudra envoyer l'adresse I2C du module suivit de la commande.

\noindent Sachant que la commande est pr\'{e}c\'{e}d\'{e}e et suivit d'une Acknowledge.

\noindent L'acknowledge est g\'{e}n\'{e}rer par le module, donc si pendant le debug il n'a pas d'acknowledge cela signifie que :

\noindent  la trame n'est pas envoy\'{e}e correctement

\noindent  L'adresse I2C n'est pas correcte

\noindent  Capteur est d\'{e}fectueux

\noindent Pour visualiser la trame :\includegraphics*[width=5.44in, height=1.14in, keepaspectratio=false]{image1}

\noindent \textit{Figure 10 : Image Datasheet Trame}

\noindent 

\noindent Slave~: Capteur

\noindent Master~: Carte STM32

\noindent Chaque commande envoy\'{e}e au capteur devra \^{e}tre r\'{e}alis\'{e}e de cette fa\c{c}on.

\noindent 

<<<<<<< HEAD
\noindent \includegraphics*[width=4.14in, height=5.46in, keepaspectratio=false]{imagealgo}Algorithme global
=======
\noindent \includegraphics*[width=2.14in, height=5.46in, keepaspectratio=false]{image2}Algorithme global
>>>>>>> master

\noindent Pour r\'{e}cup\'{e}rer une valeur selon les informations auparavant, il faut faire de cette fa\c{c}on :

\noindent au d\'{e}marrage il faut remettre \`{a} z\'{e}ro, le capteur puis demander les 9 coefficients.

\noindent On envoie les r\'{e}solutions voulues au capteur. On stocke les 2 valeurs re\c{c}ues puis conversion.

\noindent La conversion est en cours de r\'{e}alisation.

\noindent 

\noindent 

\noindent 

\noindent 

\noindent 

\noindent 

\noindent 

\noindent 

\noindent 


\section{ Test du Capteur }

\noindent 

\noindent 

\noindent Lors de la mise en place du capteur, le module r\'{e}ponds \`{a} une adresse I2C non d\'{e}finie dans la datasheet , les adresses d\'{e}finies dans la datasheet ne fonctionne pas.

\noindent Le module envoie des valeurs incoh\'{e}rentes puis apr\`{e}s des semaines de test le module ne r\'{e}ponds plus. Le module est tr\`{e}s difficile \`{a} mettre en place.

\noindent 

\noindent \eject 


\chapter{ Envoie de la valeur}


\section{ Mise en place de la communication}

\noindent Apr\`{e}s avoir fait cela on met en place le protocole UART, pour l'envoie, les param\`{e}tres~:

\begin{enumerate}
\item  Asynchrone

\item  115200 baud , pour la vitesse de transmission

\item  8bits de donn\'{e}es 

\item  Pas de bits de parit\'{e}

\item  1 bits de stop

\item  Reception et transmission
\end{enumerate}

\noindent Seulement activation de l'UART2, UART2 est celui qui est reli\'{e}e \`{a} l'USB de la Shield de la STM32.

\noindent Une fois avoir mis en place l'UART avec le CUBMX

\noindent \includegraphics*[width=5.75in, height=5.48in, keepaspectratio=false]{image3}

\noindent Une fois les param\`{e}tres mise en place on g\'{e}n\`{e}re le code et toute les initialisations utile pour notre cas.

\noindent Maintenant nous pouvons utiliser les HALs 

\noindent 

\noindent 

\noindent Pour la fonction a utilis\'{e}~:

\begin{enumerate}
\item  HAL\_UART\_Transmit : pour transmettre une valeur en UART
\end{enumerate}

\noindent Dans la fonction, 4 param\`{e}tres~:

\begin{enumerate}
\item  L'uart sur lequels envoyer les donn\'{e}es

\item  Le tableau contenant les donn\'{e}es a envoy\'{e}es

\item  Le nombre d'octets ou de caract\`{e}re a envoy\'{e}s

\item  Le Timeout
\end{enumerate}

\noindent 

\noindent Pour envoyer les valeurs du capteur il faut d'abord convertir le type re\c{c}ues par le capteur~: 

\begin{enumerate}
\item  Non sign\'{e}s, 32 bits 
\end{enumerate}

\noindent Pour convertir le format de donn\'{e}es il faut r\'{e}aliser un cast.

\noindent \textbf{void} \textbf{Send\_Data}(uint32\_t data)

\noindent $\mathrm{\{}$

 \textbf{char} tab[4];

 \textbf{int} test = (\textbf{int})data;

 \textbf{itoa}(test, tab,10);

 HAL\_UART\_Transmit(\&huart2, tab, 4, 2000);

 HAL\_UART\_Transmit(\&huart2, "{\textbackslash}r{\textbackslash}n", 3, 2000);

\noindent $\mathrm{\}}$

\noindent Une fois la valeur convertit au bon format de donn\'{e}es, il faut convertit le type Int to Char, autrement dit types entiers vers un type caract\`{e}res.

\noindent Pour cela on utilise la fonction de la librairie $\mathrm{<}$stdio$\mathrm{>}$, pour utiliser la fonction itoa, avec un le premi\`{e}re param\`{e}tre c'est la valeur enti\`{e}re \`{a} convertir en caract\`{e}res, le deuxi\`{e}me param\`{e}tre est le tableau de caract\`{e}res et le dernier param\`{e}tres et le format de donn\'{e}e que l'ont veut avoir autrement dit d\'{e}cimale, binaire, hexad\'{e}cimale, etc {\dots}

\noindent On envoie avec la fonction HAL pr\'{e}senter ci-dessus en mettant en param\`{e}tre le tableau contenant les caract\`{e}res convertit du type entier.

\noindent 


\section{ Test de la communication}

\noindent 

\noindent Pour tester l'envoie, comme le capteur ne fonctionne pas correctement est tr\`{e}s difficile \`{a} mettre en place je ferait varier une valeur du m\^{e}me type que le capteur. De 0 \`{a} 100 (comme la temperature). Et on envoie les donn\'{e}es.

\noindent \textbf{void} \textbf{TestSend\_Data}()

\noindent $\mathrm{\{}$

 uint32\_t RandomValue = 0;

\noindent 

 \textbf{for} (RandomValue = 0; RandomValue $\mathrm{<}$ 99; ++RandomValue) $\mathrm{\{}$

\noindent 

  Send\_Data(RandomValue);

 $\mathrm{\}}$

\noindent $\mathrm{\}}$

\noindent 

 On teste ce qui est re\c{c}ue par le p\'{e}riph\'{e}rique.

\noindent Pour le teste, on utilise putty avec les param\`{e}tres suivants~:

\begin{enumerate}
\item  Le COM de la shield de la STM32

\item  Vitesse~: 115200 Baud
\end{enumerate}
